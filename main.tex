% Auth: Nicklas Vraa
% Docs: https://github.com/NicklasVraa/LiX
% Everything you need to know about this template is found in on the github repository above. Stars are very appreciated.

\documentclass{novel}

%\size{a5}

\newcommand{\gamename}{Meadows and Mischiefs}

\lang      {english}
\title     {\gamename}
\subtitle  {The Role Playing Game}
\authors   {Jonathan Beechner}
\cover     {resources/novel_front.pdf}{resources/novel_back.pdf}
\license   {CC}{by-nc-sa}{3.0}
\isbn      {978-0201529838}
\publisher {NV Publishing Company}
\edition   {1}{2023}
\dedicate  {Josiah and Eliza}{For endless adventures}
\thank     {Thank you to me for being the best}
\keywords  {fiction, template, packages}

\note{Lorem ipsum dolor sit amet, consectetur adipiscing elit. Praesent porttitor est arcu, sed euismod metus imperdiet iaculis. Quisque vestibulum molestie nulla, non consectetur tellus mollis a. Nunc commodo magna a elit commodo dignissim.}

\blurb{Maecenas urna nisi, luctus nec lorem eu, vehicula varius eros. Nullam non quam tempus, ultrices lorem at, viverra felis. Nam eu ligula sodales, suscipit sem sed, bibendum augue.}

\begin{document}

\toc

\h{Whiskers in the Wheat}
\l{T}he following is an introductory scenario for the \gamename{} role playing game. This adventure showcases the basic rules and serves as an orientation both to new players (young and old) as well as new game masters.

The adventure is written for four players but could easily accommodate more or fewer. The Lark is available to play as a bonus fifth character.

The players will take on the role of four mice scouts, who have embarked on a mission to find a new source of food for their mischief. Their current home is a berry bush. As the GM, you may pick any number of reasons the mice want to leave their home. Maybe they are being encroached on by rats. Perhaps the berries are going out of season. Or maybe the bush is dying.

\hh{The Scouting Party}
Below is a description of the four mouse scouts. In addition to the details provided, each mouse wears a leaf poncho and carries a satchel containing their supplies and some berries for food. The players are responsible for choosing which mouse they would like to play and for assigning a name to their character (example names are provided).

\hhh{The Forager}
Carries a slingshot, 8 small stones, a twine rope, and a water shell. Wears an acorn cupule as a stylish hat. Responsible for finding viable sources of food and protecting against bird attacks.

\hhh{The Navigator}
Carries a portable shelter, a staff, various navigation tools, a flint and steel, and an old snail shell filled with oil for use as a lamp. Responsible for navigating the party, reading the weather, and making camp.

\hhh{The Medic}
Wields poison blow darts (6 darts). Carries a tourniquet, aromatic root, and bandages. Responsible for preventing death in the party.

\hhh{The Soldier}
Carries a spear. Wears an acorn pericarp that has been hollowed and fashioned into a helmet. Also wears light woven clothing for extra protection. Primarily responsible for the protection of the party.

\hh{The Weasel and the Lark}
\hhh{The Weasel}
The Weasel wears a vest and an eye patch and carries a dagger. He does not initially see the players, so they may have a surprise round if they intend to initiate combat. The weasel should be aggressive towards the characters and attack them if necessary. If the players make a coordinated attack, then the weasel will throw his dagger at the nearest player and then attempt to flee. If the players are uncoordinated (for example, one player attempts to take on the weasel alone), then he will attempt to kill that player as long as he is not outnumbered. The weasel is a full size larger than the mice, so players should make attack rolls with advantage.

STR: 3d6, DEX: 4d6 (highest 3), CON: 3d6, HP: 2d8, Size: -1, Dagger: 1d6 piercing

\hh{A Trepid Crossing}

\hh{Golden Fields}
When the players arrive at the wheat fields, remind them that they are probably low on food and may need a resupply before heading home, not to mention the elder mice will want proof of the group's findings.

Encourage the players to explain how they intend on gathering wheat. Remind them that the stalks of wheat tower above the mice's heads. A sample solution is that one mouse climbs a stalk, breaks off heads of wheat, and throws them down to his friends below.

\hhh{A Predator Nearby}

\hh{Respite at Last}



\h{Ipsum}
Lorem ipsum dolor sit amet, consectetur adipiscing elit. Phasellus ornare pulvinar dolor, blandit ornare sapien lacinia et. Proin erat orci, molestie sed eros at, interdum egestas sem. Nulla facilisi. Phasellus suscipit porttitor augue eu finibus. Morbi eget metus nec ex venenatis vehicula. Vivamus nec eros accumsan, mattis nisi at, auctor diam. Vestibulum nec purus dui.

Curabitur porttitor, leo eu bibendum pellentesque, lacus urna elementum nisi, in faucibus nisl lectus a libero. Fusce dictum, nisi et interdum ultrices, eros nisl scelerisque risus, eu dictum felis quam eget leo. Etiam egestas tellus pretium erat blandit, dictum tempor lacus egestas. Nunc egestas, neque sit amet venenatis efficitur, velit velit efficitur nibh, at ultrices dui lorem ut lectus. In hac habitasse platea dictumst. Nunc aliquet nisl eget ex mattis fermentum. Praesent congue lacus quis dapibus pretium. Duis auctor risus massa, vitae venenatis ligula condimentum nec.

Vestibulum ante ipsum primis in faucibus orci luctus et ultrices posuere cubilia curae; Etiam placerat ut quam eget vulputate. Morbi pulvinar quis neque sed porttitor. Donec quis malesuada arcu. Aliquam ultrices lacus quis massa convallis ultrices sed sit amet arcu. Cras ac pulvinar ante. Pellentesque habitant morbi tristique senectus et netus et malesuada fames ac turpis egestas. Nullam posuere, arcu ut ornare ultrices, odio ligula malesuada metus, nec dignissim odio urna a elit. Praesent id felis eget leo interdum accumsan dapibus non lorem. Phasellus suscipit justo sed mi vestibulum, et pretium purus sollicitudin. Etiam fringilla mollis imperdiet. Nam consectetur, lorem a hendrerit vehicula, arcu metus eleifend eros, et pulvinar enim elit nec dui. Donec non viverra est.

In massa sem, dictum eu elit ut, pretium facilisis libero. Fusce mauris ipsum, lacinia et posuere nec, dapibus id nulla. Integer cursus orci eu nunc pulvinar, at mollis est dignissim. Etiam pellentesque lobortis augue, vitae pulvinar tortor volutpat eu. Quisque elementum blandit dolor, ac egestas turpis laoreet ac. Ut tristique imperdiet felis, in pulvinar magna auctor ac. Aliquam vehicula in mauris vel tempus. Vivamus mollis tellus vestibulum, semper dolor eget, laoreet purus. Suspendisse mauris diam, tempor non fringilla vitae, fringilla at urna. Fusce aliquet, augue ac fermentum tempus, enim lectus sollicitudin nisl, sit amet posuere massa odio eu mi. Integer non orci purus. Sed sapien nisi, egestas sed ligula et, tempus luctus odio. In feugiat dolor sed sapien laoreet tincidunt. Phasellus congue metus ante, sit amet tristique nisi interdum at. Cras eget enim nisl. Orci varius natoque penatibus et magnis dis parturient montes, nascetur ridiculus mus.

\end{document}
